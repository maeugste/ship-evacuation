\documentclass[11pt]{article}
\usepackage{geometry}                
\geometry{letterpaper}                   

\usepackage{graphicx}
\usepackage{amssymb}
\usepackage{epstopdf}
\usepackage{natbib}
\usepackage{amssymb, amsmath}
\DeclareGraphicsRule{.tif}{png}{.png}{`convert #1 `dirname #1`/`basename #1 .tif`.png}

%\title{Title}
%\author{Name 1, Name 2}
%\date{date} 

\begin{document}



\thispagestyle{empty}

\begin{center}
\includegraphics[width=5cm]{ETHlogo.eps}

\bigskip


\bigskip


\bigskip


\LARGE{ 	Lecture with Computer Exercises:\\ }
\LARGE{ Modelling and Simulating Social Systems with MATLAB\\}

\bigskip

\bigskip

\small{Project Report}\\

\bigskip

\bigskip

\bigskip

\bigskip


\begin{tabular}{|c|}
\hline
\\
\textbf{\LARGE{Modeling of a passenger ship evacuation}}\\
\textbf{\LARGE{}}\\
\\
\hline
\end{tabular}
\bigskip

\bigskip

\bigskip

\LARGE{Manuela Eugster \& Andreas Reber \& Raphael Brechbuehler \& Fabian Schmid }



\bigskip

\bigskip

\bigskip

\bigskip

\bigskip

\bigskip

\bigskip

\bigskip

Zurich\\
November 2012\\

\end{center}



\newpage

%%%%%%%%%%%%%%%%%%%%%%%%%%%%%%%%%%%%%%%%%%%%%%%%%

\newpage
\section*{Agreement for free-download}
\bigskip


\bigskip


\large We hereby agree to make our source code for this project freely available for download from the web pages of the SOMS chair. Furthermore, we assure that all source code is written by ourselves and is not violating any copyright restrictions.

\begin{center}

\bigskip


\bigskip


\begin{tabular}{@{}p{3.3cm}@{}p{6cm}@{}@{}p{6cm}@{}}
\begin{minipage}{3cm}

\end{minipage}
&
\begin{minipage}{6cm}
\vspace{2mm} \large Manuela Eugster
\end{minipage}

\begin{minipage}{6cm}
\vspace{2mm} \large Raphael Brechbuehle
\end{minipage}
&
\begin{minipage}{6cm}
\vspace{2mm}\large Andreas Reber
\end{minipage}

\begin{minipage}{6cm}
\vspace{2mm} \large Fabian Schmid

\end{minipage}
\end{tabular}


\end{center}
\newpage

%%%%%%%%%%%%%%%%%%%%%%%%%%%%%%%%%%%%%%%



% IMPORTANT
% you MUST include the ETH declaration of originality here; it is available for download on the course website or at http://www.ethz.ch/faculty/exams/plagiarism/index_EN; it can be printed as pdf and should be filled out in handwriting


%%%%%%%%%% Table of content %%%%%%%%%%%%%%%%%

\tableofcontents

\newpage

%%%%%%%%%%%%%%%%%%%%%%%%%%%%%%%%%%%%%%%



\section{Abstract}

\section{Individual contributions}
The whole project was completed as a team. For sure we took into consideration all the personal backgrounds and  knowledge. That is the reason why Raphael and Manuela focused on implenting the computer code. Whereas Andreas and Fabian concentrated on providing background information, compared the results with the reality and doing its verification.
\section{Introduction and Motivations}
\subsection{Introduction}
The evacuation of a passenger liner due to fire, sinking or other issues leads to several problems. A large amount of passengers try to safe their lives and get to a rescue boat. Narrow and branched floors, smoke, inflowing water, the absence of illumination, rude passengers and so forth can make the evacuation difficult and reduce the number of survivors.
There are a lot of norms how to minimize the harm of such an evacuation. For example there are rules on the number of rescue boats dependent on the amount of passengers [2]. With dry runs the staff is prepared for the case of emergency et cetera. In real life ship corridor reproduction, the behavior of distressed people is studied.
Another approach is to model such ship evacuations numerically on the computer. As an example the software maritimeEXODUS by a development team from the University of Greenwich is a PC based evacuation and pedestrian dynamics model that is capable of simulating individual people, behaviour and vessel details. The model includes aspects of people-people, people-structure and people-environment interaction. It is capable of simulating thousands of people in very large ship geometries and can incorporate interaction with fire hazard data such as smoke, heat and toxic gases and angle of heel.” [5]
Our approach is similarly to model a passenger ship with a common geometrical outline and ground view. In an optimization process we will thereafter look for an ideal ground view, rescue boat distribution and their size to minimize the time needed for evacuation. Finally we will make a statement on possible improvements.
\newpage
\subsection{Motivation}
Even though modern ocean liners are considered to be safe, the latest occasions attested that there is still potential for evacuation and safety improvements. Certainly we know that this science is very advanced and practised since the sinking of the Titanic. Nevertheless knowing that there are still bottlenecks on the ships we are very motivated to detect and eliminate them with our mathematical models.

\section{Description of the Model}

\section{Fundamental Questions}
To find these bottlenecks we run a mathematical model of a ship structure with several decks and its passengers. After we localised these places we are interested in the answers of the folowing questions:
\section{Implementation}

\section{Simulation Results and Discussion}

\section{Summary and Outlook}

\section{References}






\end{document}  



 
